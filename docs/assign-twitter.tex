\documentclass[]{article}
\usepackage{lmodern}
\usepackage{amssymb,amsmath}
\usepackage{ifxetex,ifluatex}
\usepackage{fixltx2e} % provides \textsubscript
\ifnum 0\ifxetex 1\fi\ifluatex 1\fi=0 % if pdftex
  \usepackage[T1]{fontenc}
  \usepackage[utf8]{inputenc}
\else % if luatex or xelatex
  \ifxetex
    \usepackage{mathspec}
  \else
    \usepackage{fontspec}
  \fi
  \defaultfontfeatures{Ligatures=TeX,Scale=MatchLowercase}
\fi
% use upquote if available, for straight quotes in verbatim environments
\IfFileExists{upquote.sty}{\usepackage{upquote}}{}
% use microtype if available
\IfFileExists{microtype.sty}{%
\usepackage{microtype}
\UseMicrotypeSet[protrusion]{basicmath} % disable protrusion for tt fonts
}{}
\usepackage[margin=1in]{geometry}
\usepackage{hyperref}
\hypersetup{unicode=true,
            pdftitle={Twitter assignment},
            pdfborder={0 0 0},
            breaklinks=true}
\urlstyle{same}  % don't use monospace font for urls
\usepackage{graphicx,grffile}
\makeatletter
\def\maxwidth{\ifdim\Gin@nat@width>\linewidth\linewidth\else\Gin@nat@width\fi}
\def\maxheight{\ifdim\Gin@nat@height>\textheight\textheight\else\Gin@nat@height\fi}
\makeatother
% Scale images if necessary, so that they will not overflow the page
% margins by default, and it is still possible to overwrite the defaults
% using explicit options in \includegraphics[width, height, ...]{}
\setkeys{Gin}{width=\maxwidth,height=\maxheight,keepaspectratio}
\IfFileExists{parskip.sty}{%
\usepackage{parskip}
}{% else
\setlength{\parindent}{0pt}
\setlength{\parskip}{6pt plus 2pt minus 1pt}
}
\setlength{\emergencystretch}{3em}  % prevent overfull lines
\providecommand{\tightlist}{%
  \setlength{\itemsep}{0pt}\setlength{\parskip}{0pt}}
\setcounter{secnumdepth}{0}
% Redefines (sub)paragraphs to behave more like sections
\ifx\paragraph\undefined\else
\let\oldparagraph\paragraph
\renewcommand{\paragraph}[1]{\oldparagraph{#1}\mbox{}}
\fi
\ifx\subparagraph\undefined\else
\let\oldsubparagraph\subparagraph
\renewcommand{\subparagraph}[1]{\oldsubparagraph{#1}\mbox{}}
\fi

%%% Use protect on footnotes to avoid problems with footnotes in titles
\let\rmarkdownfootnote\footnote%
\def\footnote{\protect\rmarkdownfootnote}

%%% Change title format to be more compact
\usepackage{titling}

% Create subtitle command for use in maketitle
\newcommand{\subtitle}[1]{
  \posttitle{
    \begin{center}\large#1\end{center}
    }
}

\setlength{\droptitle}{-2em}

  \title{Twitter assignment}
    \pretitle{\vspace{\droptitle}\centering\huge}
  \posttitle{\par}
    \author{}
    \preauthor{}\postauthor{}
    \date{}
    \predate{}\postdate{}
  

\begin{document}
\maketitle

\subsection{Introduction}\label{introduction}

Scientists frequently use Twitter to communicate about their research to
the public, discuss problems they are having with their work, and
interact with each other socially (Bombaci et al 2015, English 2014).
This is especially true among R-users, and a vibrant, inclusive
community as been built. As part of this course you will be using
Twitter to find resources for solving problems in R, as well as asking
and answering questions about your work. In this class we will use the
hashtags \#learnR and \#learnstats to flag our tweets related to the
course. I will regularly tweet items with these hashtags that I think
are relevant to the class, your interests, or to science in general.

As part of this course you will be required to use Twitter at least once
each week during the semester; this can easily be accomplished during
class. Yes, I am asking you to use social media \emph{during} class.

\subsection{Assignment tasks}\label{assignment-tasks}

There are 6 specific tasks you must complete for this ongoing
assignment.

\begin{enumerate}
\def\labelenumi{\arabic{enumi}.}
\tightlist
\item
  {[} {]} Create a Twitter account for this course (You are free to use
  a personal account; I can show you how to make a parallel account
  using your existing email address if you want. You can also create
  email account using gmail just for this class.)
\item
  {[} {]} Identify at least 1 member of the R twitter community, follow
  them, and re-tweet at least 1 of their tweets to the class using the
  tag \#learnR, including me @lobrowR
\item
  {[} {]} Tweet out the link to your ``Reading reinforcement - journal
  article'' assignment
\item
  {[} {]} Tweet out the link to your ``Reading reinforcement - web
  source'' assignment
\item
  {[} {]} Tweet at least 1 question about R to me @lobrowR and include
  \#rstats and \#learnR so the rest of the class can see it
\item
  {[} {]} Respond to or answer at least 1 question from another student
  to the class.
\end{enumerate}

In addition to the 5 tweets listed above, you need to tweet something
\#rstats related at least once a week each week during the semester;
basically, do one of the things on the list at least once each week.

Probably the easiest way to accomplish this is to to open Twitter during
class and tweet out any web resources you find while working. Another
way to complete the tasks would be to open up Twitter while reading the
book and Tweet out questions or vocab related to your work on other
assignments you are working on.

\subsection{Learning Goals \& Outcomes}\label{learning-goals-outcomes}

The goal of this set of assignments is to expose students to the Twitter
R community and the diveristy of resources that can be found there.
Students will gain practice ain how to locate and share useful online
resources using twitter and ask appropriate questions. The hope is that
this assignment will also foster an online learning community for this
course.

\subsection{References}\label{references}

Bombaci et al 2015. Using Twitter to communicate conservation science
from a professional conference. Conservation Biology.

English 2014. Why you should be on Twitter.
\href{https://www.britishecologicalsociety.org/wp-content/uploads/Twitter-Article.pdf}{Bulletin
of the British Ecological Society.}

O'Mahony 2014.
\href{http://www.conservationecology.org/news/twitter-as-a-tool-for-science-communication}{Twitter
As A Tool For Science Communication}

\subsection{Twitter Assigment Details}\label{twitter-assigment-details}

To complete this assignment you must carry out the following 6 tasks on
twitter. Once you have completed

\subsubsection{{[} {]} Twitter assignment Part 1: Getting
started}\label{twitter-assignment-part-1-getting-started}

\begin{enumerate}
\def\labelenumi{\arabic{enumi}.}
\tightlist
\item
  {[} {]} Establish an account on twitter.\\
\item
  {[} {]} Follow your instructor @lobrowR
\item
  {[} {]} Send a tweet to me saying ``Hi, its XXXXX, I'm on twitter!''
\end{enumerate}

You can use a personal twitter account if you already have one or make
one just for this class. If you want to make a new twitter account for
the class separate from your personal one you

\begin{enumerate}
\def\labelenumi{\arabic{enumi}.}
\tightlist
\item
  can use your university email address
\item
  make a new gmail account to use
\item
  If you already use gmail, you can have twitter associate 2 separate
  accounts with the same email address by simply adding extra periods
  within the address. For example, I have accounts associate with my raw
  email address
  (\href{mailto:brouwern@gmail.com}{\nolinkurl{brouwern@gmail.com}}) and
  also a modified address
  (\href{mailto:brouwer.n@gmail.com}{\nolinkurl{brouwer.n@gmail.com}};
  note the ``r.n''). This works because twitter sees ``brouwern'' and
  ``brouwer.n'' as 2 seperate addresses, while Google treats them the
  same; nothing has to be done to gmail to get this to work.
\end{enumerate}

\subsubsection{{[} {]} Twitter assignment Part 2: Be a
follower}\label{twitter-assignment-part-2-be-a-follower}

\begin{enumerate}
\def\labelenumi{\arabic{enumi}.}
\tightlist
\item
  {[} {]} Find and follow at least 1 scientist, R user, or R or
  stats-related twitter account
\item
  {[} {]} Monitor this person and when they tweet something cool, send
  it to me and the rest of the class by including @lobrowR and at least
  one applicable tag: \#learnR, \#learnstats or, \#rstats.
\item
  {[} {]} You can continue to re-tweet material from interesting people
  throughout the semester to fulfill ``tweet-a-week'' aspect of the
  assignment
\end{enumerate}

Recommended accounts include

\begin{itemize}
\tightlist
\item
  Mara Averick: @dataandme
\item
  R4DS online learning community: @R4DScommunity ``Twitter home of the
  \#R4DS Online Learning Community (inspired by the R for Data Science
  text). 😍 \#rstats \& \#DataScience. Join us:
  \url{https://bit.ly/2qIxlUo}
\item
  R-Ladies Global: @RLadiesGlobal ``\ldots{}Promoting diversity in the
  \#rstats community via meetups, mentorship \& global collaboration!
  90+ groups worldwide. \#RLadies''
\item
  @rstatsbot1234 ``I retweet \#rstats stuff. Made by @shahronak47''"
\item
  Dr Andrew MacDonald: @polesasunder ``What determines where organisms
  are found \& what do they do once present?''
\item
  Carl Boettiger: @cboettig ``Professor, UC Berkeley. Theoretical
  ecology \& evolution, data science, open science. Co-founder
  @rOpenSci''
\end{itemize}

\subsubsection{{[} {]} Twitter assignment Parts 3 \& 4: Sharing your web
*and\& peer reviewed article
resources}\label{twitter-assignment-parts-3-4-sharing-your-web-and-peer-reviewed-article-resources}

\begin{enumerate}
\def\labelenumi{\arabic{enumi}.}
\tightlist
\item
  {[} {]} Complete the web resource and journal article Reading
  Reinforcement assignments
\item
  {[} {]} Tweet to me @lobrowR and include applicable tags so the rest
  of the class can find it (\#learnR, \#learnstats, \#rstats)
\end{enumerate}

Your tweet could look something like

\begin{quote}
``Found this interesting article in Animal Behavior on the use of
p-values. `Are significance thresholds appropriate for the study of
animal behavior' by Andrew Soehr 1999.
\url{https://www.sciencedirect.com/science/article/pii/S0003347298910168?via\%3Dihub}''
\end{quote}

\subsubsection{{[} {]} Twitter assignment Part 5: Ask a
question}\label{twitter-assignment-part-5-ask-a-question}

\begin{enumerate}
\def\labelenumi{\arabic{enumi}.}
\tightlist
\item
  {[} {]} Find something in R, statistics, the book, etc you don't
  understand
\item
  {[} {]} Tweet me @lobrowR; include applicable tags (\#learnr,
  \#learnstats, \#rstats, \#ecology)
\item
  {[} {]} You can continue to tweet questions to me (or other people in
  the class) throughout the semester to fulfill ``tweet-a-week'' aspect
  of the assignment, just include applicable tags (\#learnr,
  \#learnstats, \#rstats, \#ecology)
\end{enumerate}

A really good thing to tweet would be a screen grab \#rstats code that
isn't working and write something like

\begin{quote}
``hey @lobrowr, HELP!! This isn't working \#learnR.'' {[} include screen
grab; ideally circle the thing that is particularly broken{]}
\end{quote}

or another good one would be

\begin{quote}
``Um, @lobrowr the code for this week's assignment totally isn't
working. Can u fix it?''
\end{quote}

You can also ask me something stats related from a paper or work in
another class; include a screen shot of relevant text or graphs; eg

\begin{quote}
``hey @lobrowr is there something goofy about this graph? {[}paste
screen clip{]} shouldn't the error bars be asymmetric? \#learnstats
\#ecology''
\end{quote}

\subsubsection{{[} {]} Twitter assignment part 6: answer a
question}\label{twitter-assignment-part-6-answer-a-question}

\begin{enumerate}
\def\labelenumi{\arabic{enumi}.}
\tightlist
\item
  {[} {]} Respond using to at least one question someone in the class
  has posted. You don't even need to be right, just give you best guess!
\item
  {[} {]} You can continue to respond to questions to meet the
  ``tweet-a-week'' goal.
\end{enumerate}

A response could be

\begin{quote}
``hey {[}handle of person in class{]} I think you code doesn't work b/c
you are missing a comma after data = df. LOL''
\end{quote}

\subsubsection{{[} {]} Twitter assignment parts 7-15: other
contributions}\label{twitter-assignment-parts-7-15-other-contributions}

\begin{enumerate}
\def\labelenumi{\arabic{enumi}.}
\tightlist
\item
  Each week tweet out something related to your class assignments,
  readings, work in class, stats, stats related to your work outside of
  class, etc.
\end{enumerate}

Your tweets could look something like

\begin{quote}
``Found this really cool ggplot code for {[}something cool{]}. check it
out at {[}website{]}. Was made by {[}twitter handle of creater, if they
use twitter{]} \#rstats \#learnr''
\end{quote}

or if you read something cool in a paper

\begin{quote}
``Read this cool paper where they made cool graphs in ggplot and shared
all the code online {[}link to paper and code{]} \#rstats \#learnr''
\end{quote}

If you aren't sure, just tweet it out - it probably would be ok.


\end{document}
